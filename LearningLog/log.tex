\documentclass{article}
\usepackage[utf8]{inputenc}

\title{Learning Log - 2AA4}
\author{Joshua Guinness, guinnesj, 400134735}
\date{January - April 2019}

\begin{document}

\maketitle

The way this document is structured is that each section refers to a given week within the associated dates. At the start of each section will be an overview of what I learned that week through lecture. The associated subsections will talk more in depth about specific work on assignments or studying for midterm/exam that I did on a certain days. These specific subsections will also talk about struggles I had, problems I solved, and milestones I reached.

\section{Week 1, Jan 7 - 13}

This week in class I really enjoyed learning more in depth about the software engineering profession. Specifically our call and duty as software engineers to society and the skills that we need to have and develop to be successful. 

\subsection{Friday, January 11}
\begin{itemize}
    \item Attended tutorial and learned basic git commands and how to use doxygen
    \item Downloaded git
\end{itemize}

\subsection{Sunday, January 13}
\begin{itemize}
    \item Looked through the repo and the resources contained within
    \item Started a 40 min python video where it goes through the entire language
    \item Read through Assignment 1
\end{itemize}

\section{Week 2, Jan 14 - 20}

This week I enjoyed learning about the principles behind software engineering. Its incredibly exciting to learn these things because this is what a large part of the degree is about, designing quality software and these principles form the basis of how we measure quality.

\subsection{Monday, January 14}
\begin{itemize}
    \item Started to go over doxygen and tried to generate a pdf of the Box3D example
    \item Installed doxygen
\end{itemize}

\subsection{Tuesday, January 15}
\begin{itemize}
    \item Continued going through the doxygen tutorial
    \item Ran into an issue where pulling from the remote repo ended up with a blank Box3D python file on my local machine
    \item Restarted my command line and it started to work again
    \item Tried to generate the PDF, but the makefile did not work, realized it was because I did not have the correct version of doxygen installed so I installed the correct version
    \item Coded the first 3 functions of Step 1 of the assignment
    \item Took a while to get back into the habbit of coding in python but the first three functions of step 1, are tested and commented
\end{itemize}

\subsection{Wednesday, January 16}
\begin{itemize}
    \item Continued working on assignment 1
    \item Finished the first three functions of step 2 and halfway done the last function of step 2
    \item Just need to allocate the rest of people who do not have free choice
    \item Grew in my understanding of python, especially about dictionaries and lists
    \item Ran into problems with trying to figure out how to do certain things, but after doing some searching and trying/testing I figured it out
\end{itemize}

\subsection{Friday, January 18}
\begin{itemize}
    \item Finished the read allocation data in the morning and did some testing
    \item Finished programing the allocate function
    \item Test cases pass and functions are good
    \item Added doxygen commenting to functions
    \item Ran into an issue right before I was going to submit where make test did not work
\end{itemize}

\section{Week 3, Jan 21 - 27}

During class this week we started to learn about modules and MIS and how that will play a role in future specifications. I found it pretty confusing and difficult to understand all the terminology and follow along. Will definitely need to look at it on my own time. However, I can clearly see that this is important and once a specification like this is done, coding becomes much easier.

\subsection{Thursday, January 24}
\begin{itemize}
    \item Fixed the problem of make test not working by putting the folder with all my test files in the A1 directory
    \item Started writing my report
\end{itemize}

\subsection{Friday, January 25}
\begin{itemize}
    \item Finished writing the reflection on my report
    \item Really enjoyed writing the report and being able to reflect on what I worked on and how it could have been improved
\end{itemize}

\section{Week 4, Jan 28 - Feb 3}

This week the in class leanring that really stuck with me was going over ADT's. From first year, last semester, and high school experience I had a rough idea of what ADT's are, but didn't really understand them that in depth. I also learned what the difference between an ADT and an abstract object. This content is really applicable to future life and career so it is something I definitely enjoy learning about!

\section{Week 5, Feb 4 - 10}

This week in lecture we went over Generic MIS and functional programming. Functional programming is something I am definitely curious to learn more about and would like to explore in the future through languages that cater to that specifically like Haskell or Scheme.

\subsection{Sunday, February 10}
\begin{itemize}
    \item Caught up in lectures, learning about MIS
    \item Programmed the first module of assignment 2
\end{itemize}

\section{Week 6, Feb 11 - 17}

This week I enjoyed learning about object oriented design in lecture and how I can apply it to my future career.

\subsection{Monday, February 11}
\begin{itemize}
    \item Finished the programming of Assignment 2, learning all about reading formal specifications, lambda functions and writing object oriented code in python
    \item Learned how pytest works although didn't get that much of an opportunity to implement it
    \item This assignment was left until last minute so for future assignments care will be taken to not do that
\end{itemize}

\section{Week 7, Feb 18 - 24}

This was reading week. Did not work on 2AA4 this week. I enjoyed my break spending time with friends and family, as well as doing some other assignments.

\section{Week 8, Feb 25 - Mar 3}

In lecture what I found interesting was learning about module decomposition and how to take a general idea and break it up into smaller peices. This is a critical skill for engineers, especially software engineers.

\subsection{Monday, February 25}
\begin{itemize}
    \item Added 15 test cases for Assignment 2 part 1
    \item Used the test cases to test my code and partners code
    \item Finished writing the report
\end{itemize}

\subsection{Friday, March 1}
\begin{itemize}
    \item Began working on specification for Assignment 3
    \item Looking back on the Assignment 2 specification helps a lot in being able to fill in the blanks for this current assignment
    \item One of the hardest things about filling in the blanks is actually understanding what the program is supposed to be doing
\end{itemize}

\section{Week 9, Mar 4 - 10}

What I learned about in lecture this week is module guide as well as the maze tracing robot example. I especially enjoyed the example as was fun to do something else besides straight content, yet also learn about the principles being applied.

\subsection{Monday, March 4}
\begin{itemize}
    \item Continued working on filling in A3 spec, about 2/3 of the way done at this point
    \item Running into issues trying to understand what the program is doing as a whole so most of work is just reading and thinking
\end{itemize}

\subsection{Tuesday, March 5}
\begin{itemize}
    \item Started studying for midterm tomorrow
    \item Studying so far is just going over and reading lecture content, especially over content that is tricky and missed the first time around. Went through all the slides over again
    \item Also finished up the A3 spec. There were a couple of parts that weren't completely right but most of it was good
\end{itemize}

\subsection{Tuesday, March 6}
\begin{itemize}
    \item More studying for midterm today
    \item Went over some key slides again
    \item Did the two practise midterms to build confidence. Did pretty well on both of them 70-90 percent so I am feeling good for the midterm tonight
    \item Wrote the midterm
\end{itemize}

\section{Week 10, Mar 11 - 17}

\section{Week 11, Mar 18 - 24}

\section{Week 12, Mar 25 - 31}

\section{Week 13, Apr 1 - 7}

\section{Week 14, Apr 8 - 14}

\section{Week 15, Apr 15 - 21}

\end{document}

