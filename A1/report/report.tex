\documentclass[12pt]{article}

\usepackage{graphicx}
\usepackage{paralist}
\usepackage{listings}
\usepackage{booktabs}

\oddsidemargin 0mm
\evensidemargin 0mm
\textwidth 160mm
\textheight 200mm

\pagestyle {plain}
\pagenumbering{arabic}

\newcounter{stepnum}

\title{Assignment 1 Solution}
\author{Joshua Guinness, guinnesj}
\date{\today}

\begin {document}

\maketitle

In this assignment, a python program was created which automatically reads in data
from various files, then allocates first year engineering students into their
stream in order of free choice, then GPA. The program consists of three modules; 
one to read in the student, departement, and free choice data, one to allocate
the students and perform operations, and one which acts as a test driver, consisting
of test cases.

The rest of this document will outline the methodology behind the testing
and serve as a reflection on the assignment.

\section{Testing of the Original Program}

In order to talk about testing results and the methodology used to design test cases,
it is important to first talk about what was assumed while coding the program as
the test cases exists, or don't exists, based on what was assumed. 

The reason assumptions had to be taken is because the specification was not formal or
specific enough on how the interface was meant to work. For example, the specification
gave no instructions on program behaviour when the file does not exist. Either
an assumption had to be made that the file will always exists, or the potential
error had to be dealt with in the code, the later of which was done in this program.

The assumptions used while programming are as follows.

\begin{enumerate}
	\item All The names and macid of the students are correct and match those in
	other files such as freeChoice
	\item All students have a gender that is either 'male' or 'female' and the 
	gender is spelt correctly.
	\item All students have a GPA that is between 1.0 and 12.0.
	\item All students have three proper engineering streams that they would like
	to go into, spelt correctly.
	\item There is enough capacity in each department for every student with free choice
	to be allocated to their first choice
	\item There is enough capacity in each department for every student to be allocated
	to at least one of their top three choices.
	\item Every student with free choice passed first year engineering (GPA \
	geqslant 4.0).
\end{enumerate}

Based on the assumptions outline above, there were two types of test cases designed.

The first type of test cases checked for correctness of the functions. This type
included the second, third, fourth, fifth, sixth, and eighth test cases. These cases
were chosen to test the main functionality of each of the functions in CalcModule
at least once. After these test cases, it can be confirmed that the implementation
of the functions are correct. A description and rationale of each one will be 
breifly discussed. All these test cases passed.

The second test case checked to see whether the sort function sorted the students
in order of GPA correctly. It just compared to the output of the function to
a dictionary of sorted students which was known to be correct. 

The third, fourth, fifth, and sixth test cases check to see whether a correct
average was returned for males and females from two different data sets. The test
case returned false when the exact float value was compared with the output so to
solve the issue, a float value in a very small range was compared with the output.

The eighth test case confirmed that the allocate properly placed students into
their engineering streams first by free choice, then in order of GPA.

The second type of test cases checked for correct behaviour of program when weird
inputs, like missing files, or incomplete data was passed to the functions. This 
type included the first, seventh, and ninth test cases. The number of test cases 
of this type required to build a modest amount of confidence in the program depends
on the amount of assumptions. Since quite a few things were assumed, less test
cases of this type were needed. All these test cases passed after modification to the 
the code. The rationale behind them, problems uncovered while testing, and a brief
description of each test case will be discussed.

In the first test case, the functions in ReadAllocationData were tested instead of
those in CalcModule. It tested for program response when all the files were empty.
The functions responded correctly by returnning empty dictionaries or lists.

In the seventh test case, the average function was checked to see behaviour when
tried to get the average of all the females, but there are no females in the data.
Based on this test case failing for the first time, the code had to be modified to
return 'None', as well as correctly deal with the fact that there is no one of that
gender in the data.

Finally, the ninth test case checked to see whether the allocation function worked 
correctly when it was given student data and none of them had free choice. This was
tested to ensure that the function still operated correctly when one of the files
was empty.

All this list of nine test cases covered quite a bit of program functionality and
possible errors, building confidence in the robustness and correctless of it, 
there is still more testing that could have been done to verify these things. This
was what could be done in the timing given.

%Description of approach to testing.  Rationale for test case selection.  Summary
%of results.  Any problems uncovered through testing.

%Under testing you should list any assumptions you needed to make about the
%program's inputs or expected behaviour.

\section{Results of Testing Partner's Code}

A table outlining the summary of results after running the test script on the
partners code is seen below.

\begin{table}[h]
	\centering
	\begin{tabular}{lr}
		\toprule
		Test Case & Result\\
		\midrule
		1 & Pass\\
		2 & Pass\\
		3 & Pass\\
		4 & Pass\\
		5 & Pass\\
		6 & Pass\\
		7 & Fail - ZeroDivisionError\\
		8 & Pass\\
		9 & Pass\\
		\bottomrule
	\end{tabular}
	\caption{Table outlining the results of testing partner code}
	\label{Table:1}
\end{table}

As can be seen from the table all the test cases were passed except for the 
seventh one. The seventh test case check to see how the program behaved when
asked the caluclate the average GPA of all the females, yet there were no females
in the data.

The error that was gotten was a ZeroDivisionError at line 42 in the partners
CalcModule program. This line of code is:

\centerline{\textit{aveFemale = sumFemale /numFemale}}

This line of code divides the total sum by the number of females. However, if there
are zero females, the program tries to divide by zero, throwing an error, leading
the this particular test case to fail.

%Summary of results.

\section{Discussion of Test Results}

No clue bro

\subsection{Problems with Original Code}

Thought I already did this

\subsection{Problems with Partner's Code}

Didn't I do this like 4 sections ago?

\section{Critique of Design Specification}


%\newpage

\section{Answers to Questions}

\begin{enumerate}[(a)]

\item Answer for a)

\item Answer for b)

\item Even though the assignment did not require testing of ReadAllocationData, the 
first test case tested how the function within that module respond to empty files. 
It was tested because it was thought that it would be important to ensure that the
functions responsible for getting the data work properly and perform as intended

However some more test cases that could be done to build confidence in ReadAllocationData
are:

	\begin{itemize}
	\item No filenames are passed to function
	\item Filenames that are passed to function are incorrect
	\item Files that are passed to the function are empty
	\item Files that contain the correct data, but in a format that is not expected
	by the function, e.g. tabs instead of newline characters, or missing the students 
	macid
	\item Files that are in the correct format, but contain incorrect info, e.g. two 
	stream choices instead of three, or GPA greater than 12.o
	\end

Of the two modudles, CalcModule was chosen over ReadAllocationData to test because
CalcModule requires ReadAllocationData to be correct. By testing just CalcModule,
ReadAllocationData is also being tested at the same time because if the first doesn't
work, the second one wouldn't function correctly either.

Also, CalcModule does more interesthing things with the data and performs operations on it
making the test cases that would come out of it more unique and less obvious.

\item The problem with using strings for a set of finite elements is that it leaves
more room for error. Although the same functionality could be easily implemented with
strings, using integers or something related would reduces the margin of error in the
program. This new implementation would help catch errors early, rather than miss data
when performing operations later. For example, having 0 = male, and 1 = female, means
that when the data is read from the file, if it is not either a 1 or 0, an error is 
thrown and the mistake could be corrected. If they were instead represented as before,
and the mistake was allowed to propagate, when the average function is called, it would
miss the person with the spelling mistake representing their gender.

\item 

\item Answer for f)

\end{enumerate}

\newpage

\lstset{language=Python, basicstyle=\tiny, breaklines=true, showspaces=false,
  showstringspaces=false, breakatwhitespace=true}
%\lstset{language=C,linewidth=.94\textwidth,xleftmargin=1.1cm}

\def\thesection{\Alph{section}}

\section{Code for ReadAllocationData.py}

\noindent \lstinputlisting{../src/ReadAllocationData.py}

\newpage

\section{Code for CalcModule.py}

\noindent \lstinputlisting{../src/CalcModule.py}

\newpage

\section{Code for testCalc.py}

\noindent \lstinputlisting{../src/testCalc.py}

\newpage

\section{Code for Partner's CalcModule.py}

\noindent \lstinputlisting{../partner/CalcModule.py}

\newpage

\section{Makefile}

\lstset{language=make}
\noindent \lstinputlisting{../Makefile}

\end {document}
